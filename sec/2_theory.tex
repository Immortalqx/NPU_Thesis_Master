\chapter{相关理论基础}
\chaptermark{插入图表以及如何引用}

\section{表格}

使用 \href{http://www.tablesgenerator.com/}{http://www.tablesgenerator.com/} 生
成,可粘贴Excel。效果如表\ref{my-label}所示。注意表中的字号(五号)和表格宽度(
通栏)。外部请用 \lstinline`table` 环境,内部使用 \lstinline`tabularx` 即可。

\begin{table}[!h]
    \centering
    \caption{表格标题}
    \label{my-label}
    \begin{tabularx}{\textwidth}{CCCC}
        \toprule
        $A$ & $B$ & $A+B$ & $A\times B$ \\ \midrule
        1   & 6   & 7     & 6           \\
        2   & 7   & 9     & 14          \\
        3   & 8   & 11    & 24          \\
        4   & 9   & 13    & 36          \\
        5   & 10  & 15    & 50          \\ \bottomrule
    \end{tabularx}
\end{table}

\begin{table}[!h]
    \centering
    \caption{指定宽度与对齐方式}
    \label{my-label-2}
    \begin{tabularx}{\textwidth}{|P{2cm}|O{3cm}|Q{4cm}|C}
        \toprule
        \SI{2}{\centi\metre} & \SI{3}{\centi\metre} & \SI{4}{\centi\metre} & Other \\ \midrule
        1                    & 6                    & 7                    & 1     \\
        2                    & 7                    & 9                    & 2     \\
        3                    & 8                    & 11                   & 3     \\ \bottomrule
    \end{tabularx}
\end{table}

\section{插图}

请直接使用 \lstinline`figure` 环境,内部使用 \lstinline`includegraphics` 即可。
如果需要多张子图排版,请在 \lstinline`figure` 环境内部使用 \lstinline`minipage`
预先设置总的浮动体宽度,然后再使用 \lstinline`subfigure` 环境进行排版。

测试下文章内的图片引用。如\autoref{fig:example} 和\autoref{fig:example2} 所示,
这是两幅插图。在这其中\autoref{subfig:example2-subfig1} 是第一幅子图,
\autoref{subfig:example2-subfig2} 是第二幅子图。

\begin{figure}[htb]
    \centering
    \includegraphics[scale=0.2]{imgs/poster.png}
    \caption{
        这里是个普通的标题
    }
    \label{fig:example}
\end{figure}

\begin{figure}[htb]
    \centering
    \begin{minipage}[t]{0.96\textwidth}
        \centering
        \begin{subfigure}[t]{0.47\textwidth}
            \centering
            \includegraphics[scale=0.1]{imgs/poster.png}
            \caption{\label{subfig:example2-subfig1}}
        \end{subfigure}
        \begin{subfigure}[t]{0.47\textwidth}
            \centering
            \includegraphics[scale=0.1]{imgs/poster.png}
            \caption{\label{subfig:example2-subfig2}}
        \end{subfigure}
    \end{minipage}
    \caption{这里是另一个普通的标题}
    \label{fig:example2}
\end{figure}

\section{插入源代码}

这里给出一个 Hello World 的样例,如\autoref{code:hello-world} 所示。

\begin{lstlisting}[language={C++}, label={code:hello-world},
    caption={Hello World.cpp}]
#include <iostream>
using namespace std;

int main()
{
    // output "Hello World!"
    cout << "Hello World!" << endl;
    return 0;
}
\end{lstlisting}

\section{引用以及其他编写建议}

\LaTeX 提供了 \lstinline`ref` 和 \lstinline`autoref` 两种引用方式,其中前者只显
示序号,后者可以显示提示语,如“\autoref{code:hello-world}”表示引用代码,
而“\autoref{subfig:example2-subfig2}”表示引用图片的子图.为了方便引用以及作者阅读,
本人强烈建议使用 \lstinline`autoref` 来统一处理引用问题,同时在每一个
\lstinline`autoref` 添加提示语,如 \lstinline`fig` 和 \lstinline`tab` 分别表示插
图和表格。

由于 \XeLaTeX 在处理中文时,会自动在中文之间添加空格,所以请放心地在编写文档时换
行,防止某一行过长导致阅读时的不便。另外中英文之间的空格(包括命令)并未做严格限
制。本文推荐除在不影响最终成文的结果这一前提下,为保持文档的美观与易读,请自行选
择合适的编写方式。

\cleardoublepage
%%=============================================================================%